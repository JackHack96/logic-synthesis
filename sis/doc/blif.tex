%/*
% * Revision Control Information
% *
% * $Source: /users/pchong/CVS/sis/sis/doc/blif.tex,v $
% * $Author: pchong $
% * $Revision: 1.1.1.1 $
% * $Date: 2004/02/07 10:14:20 $
% *
% */
% Document Type: LaTeX
% Master File: blif.tex
%
%


\documentstyle[fullpage,times,11pt]{article}
\addtolength{\topmargin}{-13mm}
\addtolength{\textwidth}{10mm}
\addtolength{\textheight}{10mm}
\addtolength{\oddsidemargin}{-5mm}

    % define an environment .8 spacing (80% of single spacing)
    \def\pespace{\def\baselinestretch{.8}\small}
    \def\endpespace{}
%\fi

\newcommand{\BLIF}{{\sc blif}}

\newenvironment{code}%
{%
	\begin{pespace}\nopagebreak[3]\begin{verbatim}%
}%
{%
	\end{pespace}%
}%

\title{
	Berkeley Logic Interchange Format (\BLIF )
}

\author{
	University of California\\
	Berkeley
}

\begin{document}

\maketitle

The goal of \BLIF\ is to describe a logic-level hierarchical circuit in
textual form.  A circuit is an arbitrary combinational or sequential network
of logic functions.  A circuit can be viewed as a directed graph of
combinational logic nodes and sequential logic elements.  Each node has a
two-level, single-output logic function associated with it.  Each feedback
loop must contain at least one latch.  Each net (or signal) has only a
single driver, and either the signal or the gate which drives the signal can
be named without ambiguity.

In the following, angle-brackets surround nonterminals, and square-brackets
surround optional constructs.

\section{Models}

A model is a flattened hierarchical circuit.  A \BLIF\ file can contain many
models and references to models described in other \BLIF\ files.  A model is
declared as follows:
\begin{code}
    .model <decl-model-name>
    .inputs <decl-input-list>
    .outputs <decl-output-list>
    .clock <decl-clock-list>
    <command>
         .
         .
         .
    <command>
    .end
\end{verbatim}\end{code}

\begin{description}
\item {\em decl-model-name} is a string giving the name of the model.

\item {\em decl-input-list} is a white-space-separated list of strings
(terminated by the end of the line) giving the formal input terminals for the
model being declared.  If this is the first or only model, then these signals
can be identified as the primary inputs of the circuit.  Multiple {\em
.inputs} lines are allowed, and the lists of inputs are concatenated.

\item {\em decl-output-list} is a white-space-separated list of strings
(terminated by the end of the line) giving the formal output terminals for
the model being declared.  If this is the first or only model, then these
signals can be identified as the primary outputs of the circuit.  Multiple
{\em .outputs} lines are allowed, and the lists of outputs are concatenated.

\item {\em decl-clock-list} is a white-space-separated list of strings
(terminated by the end of the line) giving the clocks for the model being
declared.  Multiple {\em .clock} lines are allowed, and the lists of clocks
are concatenated.

\item {\em command} is one of:
\begin{pespace}\nopagebreak[3]\begin{tabular} {l l l}
{\verb|<logic-gate>|} & {\verb|<generic-latch>|} & {\verb|<library-gate>|} \\
{\verb|<model-reference>|} & {\verb|<subfile-reference>|} & {\verb|<fsm-description>|} \\
{\verb|<clock-constraint>|} & {\verb|<delay-constraint>|} 
\end{tabular}\end{pespace}

Each {\em command} is described in the following sections.
\end{description}

The \BLIF\ parser allows the {\em .model, .inputs, .outputs, .clock} and
{\em .end} statements to be optional.  If {\em .model} is not specified, the
{\em decl-model-name} is assigned the name of the \BLIF\ file being read.
It is an error to use the same string for {\em decl-model-name} in more than
one model.  If {\em .inputs} is not specified, it can be inferred from the
signals which are not the outputs of any other logic block.  Similarly, {\em
.outputs} can be inferred from the signals which are not the inputs to any
other blocks.  If any {\em .inputs} or {\em .outputs} are given, no
inference is made; a node that is not an output and does not fanout produces
a warning message.

If {\em .clock} is not specified (e.g., for purely combinational circuits)
there are no clocks.  {\em .end} is implied at end of file or upon
encountering another {\em .model}.

{\bf Important}: the first model encountered in the main \BLIF\ file is the
one returned to the user.  The only {\em .clock, clock-constraint,} and {\em
timing-constraint} constructs retained are the ones in the first model.  All
subsequent models can be incorporated into the first model using the {\em
model-reference} construct.

Anywhere in the file a `{\verb|#|}' (hash) begins a comment that extends to
the end of the current line.
Note that the character `{\verb|#|}' cannot be used in any signal names.
A `\verb|\|' (backslash) as the last character
of a non-comment line indicates concatenation of the subsequent line to the
current line.  No whitespace should follow the `\verb|\|'.  

Example:
\begin{code}
    .model simple
    .inputs a b
    .outputs c
    .names a b c        #    .names described later
    11 1
    .end


    #   unnamed model
    .names a b 	\
    c                   #    `\' here only to demonstrate its use
    11 1
\end{verbatim}\end{code}

Both models ``simple'' and the unnamed model describe the same circuit.

\section{Logic Gates}

A {\em logic-gate} associates a logic function with a signal in the model,
which can be used as an input to other logic functions.  A {\em logic-gate}
is declared as follows:
\begin{code}
    .names <in-1> <in-2> ... <in-n> <output>
    <single-output-cover>
\end{verbatim}\end{code}

\begin{description}
\item {\em output} is a string giving the name of the gate being defined.

\item {\em in-1, in-2, ... in-n} are strings giving the names of the inputs
to the logic gate being defined.

\item {\em single-output-cover} is, formally, an n-input, 1-output PLA
description of the logic function corresponding to the logic gate.  \{0, 1,
--\} is used in the n-bit wide ``input plane'' and \{0, 1\} is used in
the 1-bit wide ``output plane''.  The {\sc on}-set is specified with 1's in
the ``output plane,'' and the {\sc off}-set is specified with 0's in the
``output plane.''
The {\sc dc}-set is specified for primary output nodes only,
by using the construct {\em .exdc}.
\end{description}

A sample {\em logic-gate} with its {\em single-output-cover}:
\begin{code}
    .names v3 v6 j u78 v13.15
    1--0 1
    -1-1 1
    0-11 1
\end{verbatim}\end{code}

In a given row of the {\em single-output-cover}, ``1'' means the input is
used in uncomplemented form, ``0'' means the input is complemented, and
``--'' means not used.  Elements of a row are {\sc and}ed together, and then
all rows are {\sc or}ed.  

As a result, if the last column (the ``output plane'') of the {\em
single-output-cover} is all 1's, the first n columns (the ``input plane'')
of the {\em single-output-cover} can be viewed as the truth table for the
logic gate named by the string {\em output}.  The order of the inputs in the
{\em single-output-cover} is the same as the order of the strings {\em in-1,
in-2, ..., in-n} in the {\em .names} line.  A space between the columns of
the ``input plane'' and the ``output plane'' is required.

The translation of the above sample {\em logic-gate} into a sum-of-products
notation would be as follows:
\begin{code}
    v13.15 = (v3 u78') + (v6 u78) + (v3' j u78)
\end{verbatim}\end{code}
To assign the constant ``0'' to some logic gate {\verb|j|}, use the
following construct:
\begin{code}
    .names j
\end{verbatim}\end{code}
To assign the constant ``1'', use the following:
\begin{code}
    .names j
    1
\end{verbatim}\end{code}    

The string {\em output} can be used as the input to another {\em logic-gate}
before the {\em logic-gate} for {\em output} is itself defined.  

For a more complete description of the PLA input format, see {\em
espresso(5)}.

\section{External Don't Cares}

External don't cares are specified as a separate network within
a model, and are specified at the end of the model specification.
Each external don't care function, which is specified by a {\em .names}
construct, must be associated with a primary
output of the main model and specified as a function
of the primary inputs of the main model (hierarchical specification
of external don't cares is currently not supported).

The external don't cares are specified as follows:
\begin{code}
    .exdc
    .names <in-1> <in-2> ... <in-n> <output>
    <single-output-cover>
\end{verbatim}\end{code}

\begin{description}
\item {\.exdc} indicates that the following {\em .names} constructs
apply to the external don't care network.
\item {\em output} is a string giving the name of the primary output for
which the conditions are don't cares.

\item {\em in-1, in-2, ... in-n} are strings giving the names of the primary
inputs which the don't care conditions are expressed in terms of.

\item {\em single-output-cover} is an n-input, 1-output PLA
description of the logic function corresponding to the don't care
conditions for the output.
\end{description}

The following is an example circuit with external don't cares:
\begin{code}
    .model a
    .inputs x y
    .outputs j
    .subckt b x=x y=y j=j
    .exdc
    .names x j
    1 1
    .end

    .model b
    .inputs x y
    .outputs j
    .names x y j
    11 1
    .end
\end{verbatim}\end{code}

The translation of the above example into a sum-of-products
notation would be as follows:
\begin{code}
    j = x * y;
    external d.c. for j = x;
\end{verbatim}\end{code}

\section{Flip flops and latches}

A {\em generic-latch} is used to create a delay element in a
model.  It represents one bit of memory or state information.  The {\em
generic-latch} construct can be used to create any type of latch or
flip-flop (see also the {\em library-gate} section).  A {\em generic-latch}
is declared as follows:

\begin{code}
    .latch <input> <output> [<type> <control>] [<init-val>]
\end{verbatim}\end{code}

\begin{description}
\item {\em input} is the data input to the latch.

\item {\em output} is the output of the latch.

\item {\em type} is one of \{fe, re, ah, al, as\}, which correspond to
``falling edge,'' ``rising edge,'' ``active high,'' ``active low,'' or
``asynchronous.''  

\item {\em control} is the clocking signal for the latch.  It can be a {\em
.clock} of the model, the output of any function in the model, or the word
``NIL'' for no clock.

\item {\em init-val} is the initial state of the latch, which can be one of
\{0, 1, 2, 3\}.  ``2'' stands for ``don't care'' and ``3'' is ``unknown.''
Unspecified, it is assumed ``3.''
\end{description}

If a latch does not have a controlling clock specified, it is assumed that
it is actually controlled by a single global clock.  The behavior of this
global clock may be interpreted differently by the various algorithms that
may manipulate the model after the model has been read in.  Therefore, the
user should be aware of these varying interpretations if latches are
specified with no controlling clocks.

{\bf Important}: All feedback loops in a model must go through a {\em
generic-latch}.  Purely combinational-logic cycles are not allowed.

Examples:
\begin{code}
    .inputs d             # a clocked flip-flop
    .output q
    .clock c
    .latch d q re c 0
    .end

    .inputs in            # a very simple sequential circuit
    .outputs out
    .latch out in 0
    .names in out
    0 1
    .end
\end{verbatim}\end{code}

\section{Library Gates}

A {\em library-gate} creates an instance of a technology-dependent logic gate
and associates it with a node that represents the output of
the logic gate.  The logic function of the
gate and its known technology dependent delays, drives, etc. are stored with
the {\em library-gate}.  A {\em library-gate} is one of the following:

\begin{code}
    .gate <name> <formal-actual-list>
    .mlatch <name> <formal-actual-list> <control> [<init-val>]
\end{verbatim}\end{code}

\begin{description}
\item {\em name} is the name of the {\em .gate} or {\em .mlatch} to
instantiate.  A gate or latch with this name must be present in the current
working library.

\item {\em formal-actual-list} is a mapping between the formal parameters of
{\em name} (the terminals of the {\em library-gate}) and the actual
parameters of the current model (any signals in this model).  The format for
a {\em formal-actual-list} is a white-space-separated sequence of assignment
statements of the form:

\begin{code}
formal1=actual1 formal2=actual2 ...
\end{verbatim}\end{code}

All of the formal parameters of {\em name} must be specified in the {\em
formal-actual-list} and the single output of {\em name} must be the last one
in the list.

\item {\em control} is the clocking signal for the mlatch, which can be
either a {\em .clock} of the model, the output of any function in the model,
or the word ``NIL'' for no clock.

\item {\em init-val} is the initial state of the mlatch, which can be one of
\{0, 1, 2, 3\}.  ``2'' stands for ``don't care'' and ``3'' is ``unknown.''
Unspecified, it is assumed ``3.''
\end{description}

A {\em .gate} refers to a two-level representation of an arbitrary input,
single output gate in a library.  A {\em .gate} appears under a
technology-independent interpretation as if it were a single {\em
logic-gate}.

A {\em .mlatch} refers to a latch (not necessarily a D flip flop) in a
library.  A {\em .mlatch} appears under a technology-independent
interpretation as if it were a single {\em generic-latch} and possibly a
single {\em logic-gate} feeding the data input of that {\em generic-latch}.

{\em .gate}s and {\em .mlatch}es are used to describe circuits that have been
implemented using a specific library of standard logic functions and their
technology-dependent properties.  The library of {\em library-gate}s must be
read in before a \BLIF\ file containing {\em .gate} or {\em .mlatch}
constructs is read in.

The string {\em name} refers to a particular gate or latch in the library.
The names ``nand2,''
``inv,'' and ``jk\_rising\_edge'' in the following examples are
descriptive names for gates in the library.  The following \BLIF\
description:
\begin{code}
    .inputs v1 v2
    .outputs j
    .gate nand2 A=v1 B=v2 O=x  # given: formals of this gate are A, B, O
    .gate inv A=x O=j          # given: formals of this gate are A & O
    .end
\end{verbatim}\end{code}
could also be specified in a technology-independent way (assuming ``nand2''
is a 2-input {\sc nand} gate and ``inv'' is an {\sc inverter}) as follows:
\begin{code}
    .inputs v1 v2
    .outputs j
    .names v1 v2 x
    0- 1
    -0 1
    .names x j
    0 1
    .end
\end{verbatim}\end{code}
Similarly:
\begin{code}
    .inputs j kbar
    .outputs out
    .clock clk
    .mlatch jk_rising_edge J=j K=k Q=q clk 1   # given: formals are J, K, Q
    .names q out
    0 1
    .names kbar k
    0 1
    .end
\end{verbatim}\end{code} 
could have been specified in a technology-independent way (assuming
``jk\_rising\_edge'' is a JK rising-edge-triggered flip flop) as follows:
\begin{code}
    .inputs j kbar
    .outputs out
    .clock clk
    .latch temp q re clk 1     # the .latch
    .names j k q temp          # the .names feeding the D input of the .latch
    -01 1
    1-0 1
    .names q out
    0 1
    .names kbar k
    0 1
    .end
\end{verbatim}\end{code}

\section{Model (subcircuit) references}

A {\em model-reference} is used to insert the logic functions of one model
into the body of another.  It is defined as follows:
\begin{code}
    .subckt <model-name> <formal-actual-list>
\end{verbatim}\end{code}

\begin{description}
\item {\em model-name} is a string giving the name of the model being
inserted.  It need not be previously defined in this file, but should be
defined somewhere in either this file, a {\em .search} file, or a master
file that is {\em .search}ing this file.  (see {\em .search} below)

\item {\em formal-actual-list} is a mapping between the formal terminals
(the {\em decl-input-list, decl-output-list,} and {\em decl-clock-list}) of
the called model {\em model-name} and the actual parameters of the current
model.  The actual parameters may be any signals in the current model.  The
format for a {\em formal-actual-list} is the same as its format in a {\em
library-gate}.
\end{description}

A {\em .subckt} construct can be viewed as creating a copy of the logic
functions of the called model {\em model-name}, including all of {\em
model-name}'s {\em generic-latch}es, in the calling model.  The hierarchical
nature of the \BLIF\ description of the model does not have to be
preserved.  Subcircuits can be nested, but cannot be self-referential or
create a cyclic dependency.

Unlike a {\em library-gate}, a {\em model-reference} is not limited to one
output.  

The formals need not be specified in the same order as they are defined in
the {\em decl-input-list, decl-output-list,} or {\em decl-clock-list};
elements of the lists can be intermingled in any order, provided the names
are given correctly.  Warning messages are printed if elements of the {\em
decl-input-list} or {\em decl-clock-list} are not driven by an actual
parameter or if elements of the {\em decl-output-list} do not fan out to an
actual parameter.  Elements of the {\em decl-clock-list} and {\em
decl-input-list} may be driven by any logic function of the calling model.

Example: rather than rewriting the entire \BLIF\ description for a commonly
used subcircuit several times, the subcircuit can be described once and
called as many times as necessary:
\begin{code}
    .model 4bitadder
    .inputs A3 A2 A1 A0 B3 B2 B1 B0 CIN
    .outputs COUT S3 S2 S1 S0 
    .subckt fulladder a=A0 b=B0 cin=CIN     s=S0 cout=CARRY1
    .subckt fulladder a=A3 b=B3 cin=CARRY3  s=S3 cout=COUT
    .subckt fulladder b=B1 a=A1 cin=CARRY1  s=XX cout=CARRY2
    .subckt fulladder a=JJ b=B2 cin=CARRY2  s=S2 cout=CARRY3
                      # for the sake of example, 
    .names XX S1      # formal output `s' does not fanout to a primary output 
    1 1
    .names A2 JJ      # formal input `a' does not fanin from a primary input
    1 1
    .end
    
    .model fulladder
    .inputs a b cin
    .outputs s cout
    .names a b k
    10 1
    01 1
    .names k cin s
    10 1
    01 1
    .names a b cin cout
    11- 1
    1-1 1
    -11 1
    .end
\end{verbatim}\end{code}
    
\section{Subfile References}
A {\em subfile-reference} is:
\begin{code}
    .search <file-name>
\end{verbatim}\end{code}

\begin{description}
\item {\em file-name} gives the name of the file to search.
\end{description}

A {\em subfile-reference} directs the \BLIF\ reader to read in and define
all the models in file {\em file-name}.  A {\em subfile-reference} does not
have to be inside of a {\em .model}.  {\em subfile-reference}s can be
nested.  

Search files would usually be used to hold all the subcircuits referred to
in {\em model-reference}s, while the master file merely searches all the
subfiles and instantiates all the subcircuits it needs.

A {\em subfile-reference} is not equivalent to including the body of subfile
{\em file-name} in the current file.  It does not patch fragments of \BLIF\
into the current file; it pauses reading the current file, reads {\em
file-name} as an independent, self-contained file, then returns to reading
the current file.

The first {\em .model} in the master file is always the one returned to the
user, regardless of any {\em subfile-reference}s than may precede it.

\section{Finite State Machine Descriptions}

A sequential circuit can be specified in \BLIF\ logic form, as a finite
state machine, or both.  An {\em fsm-description} is used to insert a finite
state machine description of the current model.  It is intended to represent
the same sequential circuit as the current model (which contains logic), but
in FSM form.  The format of an {\em fsm-description} is:
\begin{code}
    .start_kiss
    .i <num-inputs>
    .o <num-outputs>
    [.p <num-terms>]
    [.s <num-states>]
    [.r <reset-state>]
    <input> <current-state> <next-state> <output>
              .
              .
              .
    <input> <current-state> <next-state> <output>
    .end_kiss
    [.latch_order <latch-order-list>]
    [<code-mapping>]
\end{verbatim}\end{code}

\begin{description}
\item {\em num-inputs} is the number of inputs to the FSM, which should
agree with the number of inputs in the {\em .inputs} construct for the
current model.

\item {\em num-outputs} is the number of outputs of the FSM, which should
agree with the number of outputs in the {\em .outputs} construct for the
current model.

\item {\em num-terms} is the number of ``{\verb|<|}input{\verb|>|}
{\verb|<|}current-state{\verb|>|} {\verb|<|}next-state{\verb|>|}
{\verb|<|}output{\verb|>|}'' 4-tuples that follow in the FSM description.

\item {\em num-states} is the number of distinct states that appear in
``{\verb|<|}current-state{\verb|>|}'' and ``{\verb|<|}next-state{\verb|>|}''
columns.

\item {\em reset-state} is the symbolic name for the reset state for the
FSM; it should appear somewhere in the ``{\verb|<|}current-state{\verb|>|}''
column.

\item {\em input} is a sequence of {\em num-inputs} members of \{0, 1, --\}.

\item {\em output} is a sequence of {\em num-outputs} members of \{0, 1, --\}.

\item {\em current-state} and {\em next-state} are symbolic names for the
current state and next state transitions of the FSM.

\item {\em latch-order-list} is a white-space-separated sequence of latch
outputs.

\item {\em code-mapping} is newline separated sequence of:
\begin{code}
.code <symbolic-name> <encoded-name>
\end{verbatim}\end{code}
\end{description}

{\em num-terms} and {\em num-states} do not have to be specified.  If the
{\em reset-state} is not given, it is assigned to be the first state
encountered in the ``{\verb|<|}current-state{\verb|>|}'' column.

The ordering of the bits in the {\em input} and {\em output} fields will be
the same as the ordering of the variables in the {\em .inputs} and {\em
.outputs} constructs if both an {\em fsm-description} and logic functions
are given. 

{\em latch-order-list} and {\em code-mapping} are meant to be used when both
an {\em fsm-description} and a logical description of the model are given.
The two constructs together provide a correspondence between the latches in
the logical description and the state variables in the {\em
fsm-description}.  In a {\em code-mapping}, {\em symbolic-name} consists of
a symbolic name from the ``{\verb|<|}current-state{\verb|>|}'' or
``{\verb|<|}next-state{\verb|>|}'' columns, and {\em encoded-name} is the
pattern of bits (\{0, 1\}) that represent the state encoding for {\em
symbolic-name}.  The {\em code-mapping} should only be given if both an {\em
fsm-description} and logic functions are given.  {\em .latch-order}
establishes a mapping between the bits of the {\em encoded-name}s of the
{\em code-mapping} construct and the latches of the network.  The order of
the bits in the encoded names will be the same as the order of the latch
outputs in the {\em latch-order-list}.  There should be the same number of
bits in the {\em encoded-name} as there are latches if both an {\em
fsm-description} and a logical description are specified.

If both {\em logic-gate}s and an {\em fsm-description} of the model are
given, the {\em logic-gate} description of the model should be consistent
with the {\em fsm-description}, that is, they should describe the same
circuit.  If they are not consistent there will be no sensible way to
interpret the model, which should then cause an error to be returned.

If only the {\em fsm-description} of the network is given, it may be run
through a state assignment routine and given a logic implementation.  A sole
{\em fsm-description}, having no logic implementation, cannot be inserted
into another model by a {\em model-reference}; the state assigned network,
or a network containing both {\em logic-gate}s and an {\em fsm-description}
can.

Example of an {\em fsm-description}:
\begin{code}
    .model 101	       # outputs 1 whenever last 3 inputs were 1, 0, 1
    .start_kiss
    .i 1
    .o 1
    0 st0 st0 0
    1 st0 st1 0
    0 st1 st2 0
    1 st1 st1 0
    0 st2 st0 0
    1 st2 st3 1
    0 st3 st2 0
    1 st3 st1 0
    .end_kiss
    .end
\end{verbatim}\end{code}
Above example with a consistent {\em fsm-description} and logical
description:
\begin{code}
   .model
   .inputs v0
   .outputs v3.2
   .latch    [6] v1   0
   .latch    [7] v2   0
   .start_kiss
   .i 1
   .o 1
   .p 8
   .s 4
   .r st0
   0 st0 st0 0
   1 st0 st1 0
   0 st1 st2 0
   1 st1 st1 0
   0 st2 st0 0
   1 st2 st3 1
   0 st3 st2 0
   1 st3 st1 0
   .end_kiss
   .latch_order v1 v2
   .code st0 00
   .code st1 11
   .code st2 01
   .code st3 10
   .names v0 [6]
   1 1
   .names v0 v1 v2 [7]
   -1- 1
   1-0 1
   .names v0 v1 v2 v3.2
   101 1
   .end
\end{verbatim}\end{code}

\section{Clock Constraints}

A {\em clock-constraint} is used to set up the behavior of the simulated
clocks, and to specify how clock events (rising or falling edges) occur
relative to one another.  A {\em clock-constraint} is one or more of the
following:
\begin{code}
    .cycle <cycle-time>
    .clock_event <event-percent> <event-1> [<event-2> ... <event-n>]
\end{verbatim}\end{code}

\begin{description}
\item {\em cycle-time} is a floating point number giving the clock cycle
time for the model.  It is a unitless number that is to be interpreted by
the user.

\item {\em event-percent} is a floating point number representing a
percentage of the clock cycle time at which a specific {\em .clock\_event}
occurs.  Fifty percent is written as ``50.0.''

\item {\em event-1} through {\em event-n} are one of the following:
\begin{code}
<rise-fall>'<clock-name>
(<rise-fall>'<clock-name> <before> <after>)
\end{verbatim}\end{code}
where {\em rise-fall} is either ``r'' or ``f'' and stands for the rising or
falling edge of the clock and {\em clock-name} is a clock from the {\em
.clock} construct.  The apostrophe between {\em rise-fall} and {\em
clock-name} is a seperator, and serves no purpose in and of itself.

\item {\em before} and {\em after} are floating point numbers in the same
``units'' as the {\em cycle-time} and are used to define the ``skew'' in the
clock edges.  {\em before} represents maximum amount of time before the
nominal time that the edge can arrive; {\em after} represents the maximum
amount of time after the nominal time that the edge can arrive.  The nominal
time is {\em event-percent}{\verb|%|} of the {\em cycle-time}.  In the
unparenthesized form for the {\em clock-event}, {\em before} and {\em after}
are assumed ``0.0.''
\end{description}

All events, {\em event-1 ... event-n}, specified in a single {\em
.clock\_event} are to be linked together.  A routine changing any one edge
should also modify the occurrence time of all the related clock edges.  

Example 1:
\begin{code}
    .clock clock1 clock2
    .clock_event 50.0 r'clock1 (f'clock2 2.0 5.0)
\end{verbatim}\end{code}

Example 2:
\begin{code}
    .clock clock1 clock2
    .clock_event 50.0 r'clock1
    .clock_event 50.0 (f'clock2 2.0 5.0)
\end{verbatim}\end{code}

Both examples specify a nominal time of 50{\verb|%|} of the cycle time, that
the rising edge of clock1 must occur at exactly the nominal time, and that
the falling edge of clock2 may occur from 2.0 units before to 5.0 units
after the nominal time.

In Example 1, if r'clock1 is later moved to a different nominal time by some
routine then f'clock2 should also be changed.  However, in Example 2 changing
r'clock1 would not affect f'clock2 even though they originally have the same
value of {\em event-percent}.

\section{Delay Constraints}

A {\em delay-constraint} is used to specify parameters to more accurately
compute the amount of time signals take to propagate from one point to
another in a model.  A {\em delay-constraint} is one or more of :
\begin{code}
    .area <area>
    .delay <in-name> <phase> <load> <max-load> <brise> <drise> <bfall> <dfall>
    .wire_load_slope <load>
    .wire <wire-load-list>
    .input_arrival <in-name> <rise> <fall> [<before-after> <event>]
    .default_input_arrival <rise> <fall> 
    .output_required <out-name> <rise> <fall> [<before-after> <event>]
    .default_output_required <rise> <fall>
    .input_drive <in-name> <rise> <fall>
    .default_input_drive <rise> <fall>
    .max_input_load <load>
    .default_max_input_load <load>
    .output_load <out-name> <load>
    .default_output_load <load>
\end{verbatim}\end{code}

\begin{description}
\item {\em rise}, {\em fall}, {\em drive}, and {\em load} are all floating
point numbers giving the rise time, fall time, input drive, and output load.

\item {\em in-name} is a primary input and {\em out-name} is a primary
output.

\item {\em before-after} can be one of \{b, a\}, corresponding to ``before''
or ``after,'' and {\em event} has the same format as the unparenthesized
form of {\em event-1} in a {\em clock-constraint}.

\item {\em .area} sets the area of the model to be {\em area}.

\item {\em .delay} sets the delay for input {\em in-name}.  {\em phase} is
one of ``{\sc inv},'' ``{\sc noninv},'' or ``{\sc unknown}'' for inverting,
non-inverting, or neither.  {\em max-load} is a floating point number for
the maximum load.  {\em brise, drise, bfall,} and {\em dfall} are floating
point numbers giving the block rise, drive rise, block fall, and drive fall
for {\em in-name}.

\item {\em .wire\_load\_slope} sets the wire load slope for the model.

\item {\em .wire} sets the wire loads for the model from the list of floating
point numbers in the {\em wire-load-list}.

\item {\em .input\_arrival} sets the input arrival time for the input {\em
in-name}.  If the optional arguments are specified, then the input arrival
time is relative to the {\em event}.

\item {\em .output\_required} sets the output required time for the output
{\em out-name}.  If the optional arguments are specified, then the output
required time is relative to the {\em event}.

\item {\em .input\_drive} sets the input drive for the input {\em in-name}.

\item {\em .max\_input\_load} sets the maximum load that the input {\em in-name} can handle.

\item {\em .output\_load} sets the output load for the output {\em out-name}.

\item {\em .default\_input\_arrival, .default\_output\_required,
.default\_input\_drive, .default\_output\_load} set the corresponding default
values for all the inputs/outputs whose values are not specifically set.
\end{description}

There is no actual unit for all the timing and load numbers.  Special
attention should be given when specifying and interpreting the values.  The
timing numbers are assumed to be in the same ``unit'' as the {\em
cycle-time} in the {\em .cycle} construct.
\end{document}
